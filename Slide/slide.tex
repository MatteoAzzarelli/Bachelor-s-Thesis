\documentclass{beamer}
\usepackage[english,italian]{babel}

\usepackage[T1]{fontenc}
\usepackage{textcomp}
\usepackage[utf8]{inputenc}	%caratteri accentati


\usetheme{Padova}

\title{\vspace{-0cm}Implementazione di un tool per il supporto alla valutazione automatica di progetti su GitHub}
\subtitle{\vspace{-0cm}Corso di Laurea Triennale in Informatica\\{\large Tesi di Laurea}}
\author{\vspace{-0.5cm}\centering{\hspace{-2.3cm}\textbf{Laureando}:\hspace{2.7cm}\textbf{Relatore}:\\Matteo Azzarelli\hspace{2cm}Prof. Francesco Santini}}
\date{\vspace{2.5cm}\color{black}{\footnotesize{ANNO ACCADEMICO 2017/2018}}}



\begin{document}

	\maketitle

	\begin{frame}{Outline}
		\tableofcontents
	\end{frame}

	\section{Introduzione}
	
	\begin{frame}{Introduzione}
		Perché abbiamo sviluppato questa applicazione?
		\pause
		\begin{enumerate}[<+->]
			\item Dissuadere gli studenti dal plagio.
			\item Scaricare i progetti assegnati su GitHub Classroom.
			\item Gestione di centinaia di progetti per svariati appelli all'anno.
		\end{enumerate}
	\end{frame}

	\section{GitHub}
	
	\begin{frame}{GitHub}
		Cosa è \textbf{GitHub}?
		
		\vspace{0.5cm}
		\textbf{GitHub} è un servizio di hosting per il controllo delle versioni. 
		
		\vspace{0.5cm}
		Il controllo delle versioni consente di tenere traccia delle modifiche apportate al codice sorgente del software.
		\pause
		
		\begin{block}{}
			Offre contenuti gratuiti per gli studenti \textbf{GitHub student Developer Pack}.
		\end{block}
		\begin{block}{GitHub Classroom}
			Consente agli insegnanti di assegnare compiti agli studenti, facendoli approcciare a GitHub.
		\end{block}
	\end{frame}

	
	\begin{frame}{Git}
		contenuto...
	\end{frame}
	
	\section{Lavori Correlati}
	
	\begin{frame}{Lavori correlati}
		contenuto...
	\end{frame}
	
	\section{Studio ed Implementazione}
	
	\begin{frame}{Studio ed Implementazione}
		contenuto...
	\end{frame}
	
	\section{Conclusioni}

%	\begin{frame}{Conclusioni}
%		\begin{block}{Normal block}
%			Fusce luctus venenatis felis quis semper
%		\end{block}
%
%		\begin{alertblock}{Alert block}
%			$$ E = (x_1 \vee \neg x_2 \vee \neg x_3) \wedge (x_1 \vee x_2 \vee x_4) $$
%		\end{alertblock}

%		\begin{exampleblock}{Example block}
%			Proin tincidunt, neque at tincidunt mollis
%		\end{exampleblock}
%	\end{frame}

	
	\begin{frame}{Conclusioni}
		L'applicazione è stata testata sul campo riportando risultati positivi.
		
		\vspace{0.5cm}
		\textbf{Futuri ampliamenti}:
		\begin{itemize}
			\item Download dei risultati di MOSS (MOSS li elimina dopo 14 giorni).
			\item Analisi statica dei progetti (es. Valgrind), per fornire un aiuto al docente ccon una pre-valutazione del progetto.
			\item Ottenere il nome e il cognome degli studenti dal file ReadMe delle repositories e costruire dinamicamente un gestionale degli strumenti.
		\end{itemize}
	\end{frame}

\end{document}
