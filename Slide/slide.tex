\documentclass{beamer}
\usepackage[english,italian]{babel}

\usepackage[T1]{fontenc}
\usepackage{textcomp}
\usepackage[utf8]{inputenc}	%caratteri accentati

\newcommand\tab[1][1cm]{\hspace*{#1}}	% define new command tab

\usepackage[font=scriptsize]{subcaption}
\usepackage[most]{tcolorbox}


%------------------------------------------------
%		Trasparenza liste
%------------------------------------------------
\setbeamercovered{transparent}


%------------------------------------------------
%		IMAGE DEFINITIONS
%------------------------------------------------
\usepackage{changepage}

\usepackage{pgf,tikz}
\usepackage{graphicx}

\pgfdeclareimage[height=5cm]{PlagiarismLevels}{./images/PlagirismLevels}
\pgfdeclareimage[height=5cm]{UMLMVC}{./images/Implementazione/MVCInterfaces}
\pgfdeclareimage[height=7cm]{GitHubClasses}{./images/Implementazione/GitHubClasses}
\pgfdeclareimage[height=6.5cm]{InterfacciaGrafica}{./images/Implementazione/InterfacciaGrafica}
\pgfdeclareimage[height=4cm]{MVCClass}{./images/Implementazione/MVC/MVCClass}
\pgfdeclareimage[width=12.8cm]{MOSSResults}{./images/MOSS/MOSSResults}
\pgfdeclareimage[width=12.8cm]{MOSSResultMatch}{./images/MOSS/MOSSResultMatch}
\pgfdeclareimage[width=10cm]{fingerprint01}{./images/MOSS/Fingerprint/fingerprint01}
\pgfdeclareimage[width=10cm]{fingerprint02}{./images/MOSS/Fingerprint/fingerprint02}
\pgfdeclareimage[width=8cm]{fingerprint03}{./images/MOSS/Fingerprint/fingerprint03}


\usetheme{Padova}

\title{Implementazione di un tool per il supporto alla valutazione automatica di progetti su GitHub}
\subtitle{\vspace{-0cm}Corso di Laurea Triennale in Informatica\\{\large Tesi di Laurea}}
\author{%\vspace{-0.5cm}\textbf{Laureando}:\\
	Matteo Azzarelli}
\date{\vspace{2.5cm}\color{black}{\footnotesize{ANNO ACCADEMICO 2017/2018}}}



\begin{document}

	\maketitle

	\begin{frame}{Outline}
		\tableofcontents
	\end{frame}

	\section{Introduzione}
	
	\begin{frame}{Introduzione}
		Perché abbiamo sviluppato questa applicazione?
		\pause
		\begin{enumerate}[<+->]
			\item Dissuadere gli studenti dal plagio.
			\item Scaricare i progetti assegnati su GitHub Classroom.
			\item Gestione di centinaia di progetti per svariati appelli ed esercitazioni all'anno, fornendo inoltre un'analisi su eventuali similitudini.
			
		\end{enumerate}
	\end{frame}

	\section{GitHub}
	
	\begin{frame}{GitHub}
		Cosa è \textbf{GitHub}?
		
		\vspace{0.5cm}
		\textbf{GitHub} è un servizio di hosting per il controllo delle versioni. 
		
		\vspace{0.5cm}
		Il controllo delle versioni consente di tenere traccia delle modifiche apportate al codice sorgente del software.
		\pause
		
		\begin{alertblock}{GitHub Classroom}
			Consente ai docenti di assegnare compiti agli studenti, facendoli approcciare a GitHub.
		\end{alertblock}
	\end{frame}
	

	\begin{frame}{Git}
		GitHub è basato sul software \textbf{Git}, il quale permette di:
		\begin{itemize}[<+->]
			\item Tenere traccia dei cambiamenti dei file.
			\item Coordinare il lavoro di più persone sullo stesso insieme di file.
			\item Mantenere l'integrità dei files.
			\item Gestire flussi di lavoro non lineari.
		\end{itemize}
	\end{frame}
	
%	\begin{frame}{Git: Comandi}
%		\begin{block}{Inizializzazione repository}
%			\tab git init
%		\end{block}
%		\pause
%		\begin{block}{Clonare repository}
%			\tab git clone <url>
%		\end{block}
%		\pause
%		\begin{block}{Rendere effettive le modifiche}
%			\tab git commit -m 'Commento'
%		\end{block}
%	\end{frame}
	
%	\begin{frame}{Git: Comandi}
%		\begin{block}{Prendere dati da remoto}
%			\tab git fetch <remote>
%		\end{block}
%		\pause
%		\begin{block}{Condividere le modifiche}
%			\tab git push <remote> <branch>
%		\end{block}
%		\pause
%		\begin{block}{Recupero dati dal Server di origine}
%			\tab git pull
%		\end{block}
%	\end{frame}

	\section{Lavori Correlati}
	
	\begin{frame}{Plagio}
		\begin{alertblock}{Definizione}
			Un programma che è stato prodotto da un altro e riportato con un numero esiguo di trasformazioni di routine.
			
			\hspace{4cm}\textit{1976. Alan Parker e James O. Hamblen.}
		\end{alertblock}
		
		\vspace{-0.5cm}
		\begin{center}
			\pgfuseimage{PlagiarismLevels}
		\end{center}
	\end{frame}

%	\begin{frame}{Lavori correlati}
%		Alcuni dei più famosi software di \textit{plagiarism detection}:
%		\begin{itemize}
%			\item[SIM] Software Similarity Tester
%			\item[Siff] Inizialmente confrontava  i file di testo. Nelle versioni più recenti hanno utilizzato le fingerprint.
%			\item[Plague] Implementazione dei suggerimenti di Whale. 
%			\item[YAP] Yet Another Plague. Lavora sempre con i token.
%			\item[JPlag] Servizio Online con un'interfaccia grafica molto funzionale. Simile a MOSS.
%		\end{itemize}
%	\end{frame}

	\section{Studio ed Implementazione}
	
	\begin{frame}{Studio ed Implementazione}
		Caratteristiche dell'applicazione:
		\begin{itemize}
			\item<1-> Sviluppata in \textbf{Java}:
				\begin{itemize}
					\item \`E cross-platform.
					\item Documentazione completa.
					\item Esistenza di librerie per il nostro scopo.
				\end{itemize}
			\item<2-> Utilizzo del pattern Model View Controller.
			\item<3-> Download delle repositories di GitHub.
			\item<4-> Analisi dei progetti degli studenti tramite il servizio MOSS.
		\end{itemize}
	\end{frame}
	
	\begin{frame}{Studio ed Implementazione: MVC}
		Diagramma UML della struttura del software:
		\begin{center}
			\pgfuseimage{UMLMVC}
		\end{center}
	\end{frame}
	
	\begin{frame}{Studio ed Implementazione: GitHub}
		\vspace{-0.2cm}
		Organizzazione delle repositories:
		\begin{center}
			\pgfuseimage{GitHubClasses}
		\end{center}
	\end{frame}
	
	\begin{frame}{Studio ed Implementazione: Interfaccia}
		\vspace{-0.2cm}
		Uno sguardo all'interfaccia:
		\begin{center}
			\pgfuseimage{InterfacciaGrafica}
		\end{center}
	\end{frame}

%	\begin{frame}{MVC}
%		Il \textbf{Model View Controller} è uno dei pattern architetturali.
%		\begin{itemize}
%			\item \textbf{Model}: modello dei dati dell'applicazione.
%			\item \textbf{View}: rappresenta la grafica e interattiva del model.
%			\item \textbf{Controller}: logica di controllo che garantisce il modello dei dati.
%		\end{itemize}
%		\begin{center}
%			\pgfuseimage{MVCClass}
%		\end{center}
%	\end{frame}
	
	\begin{frame}{MOSS}
		MOSS (Measure Of Software Similarity) è un sistema automatico per determinare la similarità di programmi.
		
		\begin{itemize}
			\item[Input] Accetta gruppi di documenti.
			\item[Output] Restituisce un'insieme di pagine HTML contenenti le coppie di documenti simili.
		\end{itemize}
		
		\pause
		Può analizzare molti linguaggi di programmazione, come ad esempio:
		\begin{itemize}
			\item \texttt{C, C++ e C\#}
			\item \texttt{Java}
			\item Il linguaggio naturale e molti altri.
		\end{itemize}
	\end{frame}

	\begin{frame}{MOSS: Funzionamento}
		Con poche fingerprint ottiene ottimi risultati. Ciò implica maggiore efficienza.
		
		Algoritmo:
			\vspace{-0.3cm}\begin{center}
				\pgfuseimage{fingerprint01}
			\end{center}
			\vspace{-0.5cm}{\footnotesize \textbf{1)} Costruisce un indice che mappa le fingerprint alle locazioni per tutti i documenti.}
	\end{frame}

	\begin{frame}{MOSS: Funzionamento}
		\vspace{-0.8cm}\begin{center}
			\pgfuseimage{fingerprint02}
		\end{center}
		\vspace{-0.5cm}{\footnotesize \textbf{2)} Per ogni documento viene effettuata una nuova fingerprint, ottenendo una lista di fingerprint per ciascun documento. Ora ogni documento \texttt{d} può contenere fingerprint di molti altri documenti \texttt{d1,d2,...}.}
	\end{frame}
	
	\begin{frame}{MOSS: Funzionamento}
		\vspace{-0.5cm}\begin{center}
			\pgfuseimage{fingerprint03}
		\end{center}
		\vspace{-0.5cm}{\footnotesize \textbf{3)} La lista delle fingerprint viene raggruppata per documento e poi vengono fatte le coppie di documenti \texttt{(d,d1), (d,d2)}.
		Queste coppie vengono ordinate per numero di fingerprint uguali.}
	\end{frame}
	
	\begin{frame}{MOSS: Fingerprint}
		
		{\footnotesize Generare le fingerprint:
		\begin{itemize}[<+->]
			\item Divide un documento in k-grams, cioè in sottostringhe di lunghezza k
			\begin{exampleblock}{\footnotesize Esempio k-gram}
				\begin{tcolorbox}[size=small]
					A do run run run, a do run run
				\end{tcolorbox}
				(a) un po di testo [The Crystals. Da do run run, 1963]
				\pause
				\begin{tcolorbox}[size=small]
					adorunrunrunadorunrun
				\end{tcolorbox}
				(b) il testo senza alcune caratteristiche irrilevanti
				\pause
				\begin{tcolorbox}[size=small]
					adoru dorun orunr runru unrun nrunr runru
					
					unrun nruna runad unado nador adoru dorun
					
					orunr runru unrun
				\end{tcolorbox}
				(c) la sequenza di 5-grams derivata dal testo
			\end{exampleblock}
			\pause
			\item Utilizza una funzione hash per ogni k-gram e selezioniamo tra essi qualche sottoinsieme, che sarà la fingerprint.
		\end{itemize}
	}
	\end{frame}

	\begin{frame}{MOSS: Servizio}
		MOSS mette a disposizione un webservice accessibile tramite vari script e librerie.
		
		Per il nostro progetto abbiamo utilizzato la libreria Java \textbf{MOJI} che consente l'accesso al servizio.
		
		\pause
		Questa libreria permette di:
		\begin{enumerate}[<+->]
			\item Connettersi al server con il proprio numero utente.
			\item Inviare la cartella contenente tutti i progetti degli studenti.
			\item Inviare la cartella contenente il modello del progetto.
			\item Ricevere il link per visualizzare la pagina HTML dei risultati.
		\end{enumerate}
	\end{frame}

	\begin{frame}{MOSS: Servizio}
		\begin{block}{Struttura cartella dei progetti}
		 	\begin{tcolorbox}
				\texttt{\tab solution\_directory\\ 
					\tab $\vert$- student1\\
					\tab[1.5cm] $\vert$- main.c\\
					\tab[1.5cm] $\vert$- ...\\
					\tab $\vert$- student2\\
					\tab[1.5cm] $\vert$- ...\\
					\tab $\vert$\_ student3\\
					\tab[1.5cm] $\vert$- ...
				}
			\end{tcolorbox}
		\end{block}
	\end{frame}
	
	\begin{frame}{MOSS: Risultati}
		\begin{adjustwidth}{-1cm}{-1cm}
			\pgfuseimage{MOSSResults}
		\end{adjustwidth}
	\end{frame}



	\begin{frame}{MOSS: Risultati}
		\begin{adjustwidth}{-1cm}{-1cm}
			\pgfuseimage{MOSSResultMatch}
		\end{adjustwidth}
	\end{frame}

	\section{Conclusioni}

	
	\begin{frame}{Conclusioni}
		Obiettivi raggiunti:
		\begin{enumerate}
			\item Scaricare i progetti assegnati su GitHub Classroom.
			\item Gestire ed analizzare centinaia di progetti, analizzandoli tramite MOSS.
		\end{enumerate}
		L'applicazione è stata testata sul campo riportando risultati positivi.
		\pause
		
		\vspace{0.5cm}
		{\small \textbf{Futuri ampliamenti}:
		\begin{itemize}
			\item Download dei risultati di MOSS (MOSS li elimina dopo 14 giorni).
			\item Analisi statica dei progetti (es. Valgrind), per fornire un aiuto al docente con una pre-valutazione del progetto.
			\item Ottenere il nome e il cognome degli studenti dal file ReadMe delle repositories e costruire dinamicamente un gestionale degli strumenti.
		\end{itemize}}
	\end{frame}

	\begin{frame}{Conclusioni}
		\begin{center}
			{\Huge \textcolor{rossoPantano}{\textbf{Grazie per l'attenzione}}}
		\end{center}
		
	\end{frame}

\end{document}
