% !TEX encoding = UTF-8
% !TEX program = pdflatex
% !TeX spellcheck = it_IT
\documentclass[11pt, a4paper, twoside, openright]{book}
\usepackage[english,italian]{babel}

\usepackage[T1]{fontenc}
\usepackage{textcomp}
\usepackage[utf8]{inputenc}	%caratteri accentati

%frontespizio
\usepackage[nowrite]{frontespizio}

%------------------------------------------------
%		INDICE
%------------------------------------------------
\usepackage[colorlinks=true,linkcolor=blue,urlcolor=black,bookmarksopen=true]{hyperref}

%------------------------------------------------
%		IMAGE DEFINITIONS
%------------------------------------------------
\usepackage{pgf,tikz}
\usepackage{graphicx}
\usepackage[font=scriptsize, labelfont=bf]{caption}

\pgfdeclareimage[width=10cm]{storingDataAsChanges}{./images/chapter01/Git/storingDataAsChanges}
\pgfdeclareimage[width=10cm]{storingDataAsSnapshots}{./images/chapter01/Git/storingDataAsSnapshots}
\pgfdeclareimage[width=10cm]{branch}{./images/chapter01/Git/branch}
\pgfdeclareimage[width=10cm]{GitHubClassroom}{./images/chapter01/Git/GitHubClassroom}

%------------------------------------------------
%	PAGE SETTINGS
%------------------------------------------------
\usepackage{fancyhdr}

\newcommand{\fncyfront}{% R: Right, L: Left, O: Odd, E: Even
	\fancyhead[RO]{{\footnotesize\rightmark}}
	\fancyfoot[RO]{\thepage}
	\fancyhead[LE]{\footnotesize{\leftmark}}
	\fancyfoot[LE]{\thepage}
	\fancyhead[RE,LO]{}
	\fancyfoot[C]{M. Azzarelli}
	\renewcommand{\headrulewidth}{0.3pt}}
\newcommand{\fncymain}{%
	\fancyhead[RO]{{\footnotesize\rightmark}}
	\fancyfoot[RO]{\thepage}
	\fancyhead[LE]{{\footnotesize\leftmark}}
	\fancyfoot[LE]{\thepage}
	\fancyfoot[C]{M.Azzarelli}
	\renewcommand{\headrulewidth}{0.3pt}}

% Redefine the plain page style (Set up dell'header e del footer delle pagine es prima pagina capitolo)
\fancypagestyle{plain}{
	\fancyhf{}%
	\fancyfoot[RO]{\thepage}
	\fancyfoot[LE]{\thepage}
	\fancyfoot[C]{M.Azzarelli}%
	\renewcommand{\headrulewidth}{0pt}% Line at the header invisible
	\renewcommand{\footrulewidth}{0.4pt}% Line at the footer visible
}

%-------------------------------------------
%		CHAPTER STYLE
%-------------------------------------------
\usepackage{titlesec, blindtext, color}
\definecolor{gray75}{gray}{0.75}
\newcommand{\hsp}{\hspace{20pt}}
\titleformat{\chapter}[hang]{\huge\bfseries}{\thechapter\hsp\textcolor{gray75}{|}\hsp}{0pt}{\Huge\bfseries}





%----------------------------------------------
%		Define abstract environment
%----------------------------------------------
\newenvironment{abstract}%
{\cleardoublepage%
	\thispagestyle{empty}%
	\null \vfill\begin{center}%
		\bfseries \abstractname \end{center}}%
{\vfill\null}

%----------------------------------------------
%		BIBLIOGRAPHY
%----------------------------------------------
\usepackage{url}	% to cite url in bibliography
\usepackage{natbib}


%----------------------------------------------
%		BOX PER COMANDI TERMINALE
%----------------------------------------------
\usepackage[most]{tcolorbox}



\begin{document}

	\fncyfront
	\pagestyle{empty}
	
	\begin{frontespizio}
		\Universita{Perugia}
		\Facolta{SCIENZE MM.FF.NN.}
		\Dipartimento{Matematica ed Informatica}
		\Corso[Laurea]{Informatica}
		\Logo[5cm]{images/Logo_unipg}
		\Titoletto{Tesi di laurea}
		\Titolo{IMPLEMENTAZIONE DI UN TOOL\\ PER IL SUPPORTO ALLA VALUTAZIONE AUTOMATICA\\ DI PROGETTI SU GITHUB }
		\NCandidato{Laureando}
		\Candidato[244999]{Matteo Azzarelli}
		\Relatore{Prof.~Francesco Santini}
		\Rientro{2.5cm}
		\Annoaccademico{2017-2018}
	\end{frontespizio}

	%----------------------------------------------
	%		Dedica
	%----------------------------------------------

	\cleardoublepage
	\null\vspace{\stretch{1}}
	\begin{flushright}
		\textit{Un ringraziamento particolare ai miei genitori\\ e a Luisa che mi hanno supportato in questi anni.}
	\end{flushright}
	\vspace{\stretch{2}}\null
	%----------------------------------------------

	%----------------------------------------------
	%	Abstract
	%----------------------------------------------
	\begin{abstract}
		Contenuto del sommario
	\end{abstract}


	\frontmatter	%impostta il layout per il menu'
	%----------------------------------------------
	%		Indici
	%----------------------------------------------
	\tableofcontents
	\listoffigures
	\listoftables
	
	\cleardoublepage
	\pagestyle{fancy}
	\fncymain	%impostta il layout per il main
	
	\mainmatter
	% !TEX root = ../tesi.tex
% !TEX encoding = UTF-8
% !TEX program = pdflatex

\chapter*{Introduzione}
\addcontentsline{toc}{chapter}{Introduzione} %aggiungo la numerazione all'indice
\chaptermark{Introduzione}

	Questa applicazione nasce dall'esigenza di trovare le similitudini tra una moltitudine di progetti di esami scritti in linguaggio \verb|C|. 
	
	Tenere sotto controllo centinaia di progetti, ricordandosi ogni singolo codice è umanamente impossibile, per questo motivo il seguente programma è un'eccellete supporto alla valutazione. Infatti il docente non ha più l'onere di memorizzare ogni codice alla ricerca di illeciti, piuttosto potrà concentrarsi sulla valutazione delle effettive capacità degli alunni.
	Lo scopo quindi è di analizzare i compiti degli studenti incrociando gli esami dello stesso appello e se necessario anche degli appelli precedenti, così da ridurre al minimo la probabilità di sfuggire al controllo.
	
	Ci auguriamo che gli studenti prendendo coscienza dell'effettiva esistenza di un software di controllo funzionante siano dissuasi dal perpetrare le azioni di copiatura. Questa teoria è anche supportata da tutti gli sviluppatori di applicazioni di plagiarism detection. 
	
	In questo progetto vedremo cosa è GitHub~(\ref{def:GitHub}) e GitHub Classroom~(\ref{def:Classroom}) con annessa introduzione al software Git~(\ref{def:Git}). Dopodiché andremo alla scoperta dei maggiori programmi di plagiarism detection~(\ref{def:PlagiarismDetection}) e in fine vedremo più in dettaglio come è stata realizzata l'applicazione~(\ref{def:StudioImplementazione}) che permette di scaricare da GitHub i progetti di esame degli studenti e in seguito di analizzare li stessi tramite il supporto di MOSS~(\ref{def:MOSS}).

	

	% !TEX root = ../tesi.tex
% !TEX encoding = UTF-8
% !TEX program = pdflatex

\chapter{GitHub Classroom}
	\section{GitHub}\label{def:GitHub}
		\textbf{GitHub} è un servizio di hosting per il controllo delle versioni basato su \textit{Git}~(\ref{def:Git}).
		Esso, oltre al controllo delle versioni, fornisce anche altre funzionalità di collaborazione come il bug tracking, gestione delle attività e wiki per ogni progetto.
		
		GitHub offre la possibilità di creare repositories sia private che pubbliche, quest'ultime spesso sono utilizzate per condividere progetti open-source, molto importante per la comunità scientifica.
		
		Nato nel 2008 questo progetto prende subito piede, infatti in nemmeno un anno raggiunge 100.000 utenti iscritti. Dopo 10 anni dalla sua nascita arriva a 22 milioni di iscrizioni diventando uno standard  e un requisito necessario per tutti i programmatori.
		
		Un importantissima iniziativa lanciata da GitHub è il nuovo programma \textbf{GitHub Student Developer Pack} che concede gratuitamente agli studenti un insieme dei tools e servizi più popolari come ad esempio \textit{awsEducate}, \textit{bitnami} o ancora \textit{Microsoft Azure} e ovviamente \textit{GitHub} offre repositories private illimitate.
		GitHub offre agli insegnanti il tool chiamato \textbf{GitHub Classroom}~(\ref{def:Classroom}), che si propone come uno strumento per aiutare gli insegnanti ad  educare le nuove generazioni di sviluppatori ad utilizzare gli strumenti richiesti dalle aziende e a padroneggiare il linguaggio necessario per immettersi nel mondo del lavoro.
	
		\subsection{Git}\label{def:Git}
			Una breve digressione sul \textit{controllo di versione distribuito \textbf{Git}}~\citep{ProGit2018}.
			
			Esso tiene traccia dei cambiamenti dei file e coordina il lavoro su questi file tra un team di più persone.
			\'E utilizzato prevalentemente per la gestione di codice sorgente di progetti di software, ma esso può essere utilizzato per tenere traccia dei cambiamenti in  qualsiasi set di files.
			
			Pensato per essere un sistema di controllo distribuito esso è stato progettato per mantenere l'integrità dei files, di garantire un'alta velocità e anche di gestire dei flussi di lavoro non lineari.
			
			Questo magnifico software è stato creato da \textit{Linus Torvalds} nel 2005 insieme ad altri suoi colleghi per facilitare lo sviluppo del kernel di Linux.
			
			Git si distingue da tutti gli altri software di versione per il modo in cui immagazzina i dati. A differenza dei precedenti VCS che salvano i dati come una lista delle modifiche (Figura~\ref{img:storingDataAsChanges}), Git salva i dati come delle istantanee (Figura~\ref{img:storingDataAsSnapshot}). 
			
			\begin{center}
				\pgfuseimage{storingDataAsChanges}
				\captionof{figure}[Salvataggio dati come cambiamenti]{Gli altri VCS tendono ad immagazzinare i dati come cambiamenti alla versione base di ogni file.}
				\label{img:storingDataAsChanges}
			\end{center}
		
			\begin{center}
				\pgfuseimage{storingDataAsSnapshots}
				\captionof{figure}[GIT salvataggio dati come snapshot]{Git immagazzina i dati come snapshot del progetto nel tempo.}\label{img:storingDataAsSnapshot}
			\end{center}
			
			Questa peculiarità rende Git simile ad un mini \textit{filesystem} con tutti i benefici di un gestore delle versioni.
			Questo modo di pensare i dati rende possibile la creazione di differenti Branch paralleli, in seguito verranno analizzati.
			
			Ora esploriamo brevemente i controlli principali di Git:
			
			\begin{itemize}
				\item Init
				\item Clone
				\item Commit
				\item Fretch
				\item Push
				\item Pull
			\end{itemize}
			Per gli altri comandi si rimanda alla guida ufficiale ProGit~\citep{ProGit2018}.
			
			\subsubsection*{Init}
				Se hai una directory di un progetto e vuoi iniziare il controllo di versione su di essa con Git dovrai utilizzare il comando:
				\begin{tcolorbox}
					\texttt{\$\qquad git init}
				\end{tcolorbox}
				Questo comando crea una nuova sottodirectory chiamata .git, la quale contiene tutti i file necessari per la repository.
				A questo punto ancora non è tenuta traccia di alcun file.
				Se vuoi iniziare il controllo di versione su file già preesistenti dovrai aggiungere i file alla repository e poi fare un commit iniziale, ad esempio aggiungiamo tutti i file \verb|.c| e un file di licenza  e facciamo il commit con un messaggio:
				\begin{tcolorbox}
					\texttt{\$\qquad git add *.c \\ \$\qquad git add LICENSE \\ \$\qquad git commit -m 'progetto versione iniziale'}
				\end{tcolorbox}
				A questo punto abbiamo una Git repository con i file tracciati e con un commit iniziale.
			
			\subsubsection*{Clone}
				Molto spesso ci troviamo a lavorare con progetti già avviati, quindi in questo caso dovremmo utilizzare il comando \verb|git clone <url>| per, appunto, clonare la repository desiderata. Esempio:
				\begin{tcolorbox}
					\texttt{\$\qquad git clone https://github.com/MatteoAzzarelli/GitHub\\ClassroomDownloader.git}
				\end{tcolorbox}
				In questo modo verrà creata una directory chiamata ClassroomDownloader, già inizializzata con la cartella .git, dove verranno scaricati tutti i file del progetto pronti per essere utilizzati.
				 
			
			\subsubsection*{Commit}
				Forse è il comando più utilizzato, in quanto permette di rendere effettive le modifiche ai file. Aggiungendo il flag \verb|-m| abbiamo la possibilità di mettere un commento al commit, così da poter capire e far capire anche ai collaboratori quale modifiche sono state apportate. Esempio:
				\begin{tcolorbox}
					\texttt{\$\qquad git commit -m 'Fix al Bug\#'}
				\end{tcolorbox}
			
			\subsubsection*{Fetch}
				Per prendere i dati dal progetto remoto, ad esempio per aggiornare i file con le modifiche apportate da un collaboratore, è possibile utilizzare il seguente comando:
				\begin{tcolorbox}
					\texttt{\$\qquad git fetch <remote>}
				\end{tcolorbox}
				dove <remote> è il nome del server remoto. si possono visualizzare i server remoti tramite il comando \verb|git remote| , il server di default è chiamato \verb|origin|.
			
			
			\subsubsection*{Push}
				Nel momento in cui vuoi condividere con gli altri collaboratori le tue modifiche, dovrai caricarle nel server i tuoi file con il comando se \texttt{git push <remote> <branch>}. Esempio:
				\begin{tcolorbox}
					\texttt{\$\qquad git push origin master}
				\end{tcolorbox}
				Il branch di default è chiamato master. Un branch è un ramo del progetto, questi spesso vengono utilizzati per aggiungere delle features al progetto, senza influire sulla versione base se la modifica non dovesse andare a buon fine. I rami possono terminare senza essere integrati nella versione base come si vede nel branch \textit{Feature 2} in Figura~\ref{img:branch}, oppure possono essere riuniti ad un altro branch come ad esempio \textit{Feature 1} o \textit{Develop} in Figura~\ref{img:branch}.
				
				\begin{center}
					\pgfuseimage{branch}
					\captionof{figure}{Esempio di un workflow a branch}
					\label{img:branch}
				\end{center}
			
			\subsubsection*{Pull}
				Il comando \verb|git pull| generalmente recupera i dati dal server da cui originariamente sono stati clonati e tenta automaticamente di unirlo al codice su cui stai attualmente lavorando. Se non dovesse riuscire ad unire i file chiederà all'utente di risolvere i conflitti generati.

		\subsection{GitHub Classroom}\label{def:Classroom}
				\begin{center}
				\pgfuseimage{GitHubClassroom}
			\end{center}
		
			\textbf{GiHub Classroom} è lo strumento messo a disposizione da GitHub per gli insegnanti. Gli step fondamentali per iniziare ad usarlo sono:
			\begin{itemize}
				\item\textbf{Nuova organizzazione}\quad Creare un'organizzazione dal pannello personale di GitHub o se già esistente sfruttare quest'ultima.
				\item\textbf{Nuova Classroom}\quad Il passo successivo è creare una nuova Classroom. Si dovrà selezionare l'organizzazione desiderata per poi inserire il nome della classe. C'è la possibilità di invitare altri insegnanti alla classe appena creata o dei membri.
				\item\textbf{Nuovo Compito}\quad A questo punto si può procedere alla creazione di un nuovo \textit{assignment} ovvero un nuovo compito.  \'E possibile fare scegliere se assegnare compiti di gruppo o individuali. Ogni compito ha un titolo, può essere reso pubblico o privato, si può anche aggiungere una repository per dare un insieme di file di base.  
			\end{itemize}
			
			
			
			
	% !TEX root = ../tesi.tex
% !TEX encoding = UTF-8
% !TEX program = pdflatex
\chapter{Lavori correlati}\label{def:PlagiarismDetection}
	Gli autori di~\citep{Parker1989} definiscono il plagio di software come "un programma che è stato prodotto da un altro e riproposto con un esiguo numero di trasformazioni di routine".
	
	Le trasformazioni che possono aver luogo possono variare da quelle più semplici (come cambiare i commenti nel codice oppure cambiare i nomi delle variabili) a quelle più complesse (rimpiazzare strutture di controllo con altre equivalenti, ad esempio sostituire un ciclo "for" con un "while"). Questo può essere visualizzato usando i sei livelli rappresentati nello spettro del plagio (figura~\ref{img:PlagirismLevels}) realizzato da \citep{Faidhi1987}.
	
	Una proprietà importante di qualsiasi sistema è la corretta individuazione di codice sospetto. 
	C'è una netta differenza tra similarità di codice nata casualmente e dovuta al plagio. 
	Questo è molto simile a individuare parole o frasi "inusuali" nel testo scritto. 
	
	\begin{center}
		\pgfuseimage{PlagirismLevels}
		\captionof{figure}{Raffigura i vari livelli di plagio all'interno di un programma}
		\label{img:PlagirismLevels}
	\end{center}

	Una fonte molto utile per la detenzione del plagio nei software è l'aticolo di \citep{Whale1990} che ha fatto una lista dei potenziali metodi o stratagemmi per camuffare il plagio, inclusi i seguenti:
	\begin{itemize}
		\item Cambiare i commenti,
		\item Cambiare il tipo di dato (es. da float a double),
		\item Cambiare il nome delle variabili,
		\item Aggiungere dichiarazioni o variabili ridondanti,
		\item Modificare le strutture degli statements di selezione (es. modificare la struttura di un if),
		\item Combinare porzioni di codice copiato ed originale.
	\end{itemize}

	Whale prosegue discutendo la valutazione dei sistemi di rilevamento dei plagi e di rilevamento della similitudine, tra cui anche il suo stesso tool chiamato Plague~(\ref{def:Plague}).

	Ora presentiamo una breve rassegna sui migliori software di \textit{plagiarism detection}.
	
	\section{SIM (Software Similarity Tester)}
		\textbf{SIM} è stato svilupppato da Dick \citep{SIM3.0.2} a Vrije Universiteit. Questo software è distribuito gratuitamente sotto forma di sorgente in linguaggio \verb|C| e testa la similarità tra programmi misurando l'affinità lessicale di testi scritti in \textit{C}, \textit{Java} , \textit{Pascal}, \textit{Modula-2}, \textit{Miranda}, \textit{Lisp}, \textit{8086 assembler code} e il linguaggio naturale.
		
		Dick Grune afferma che SIM è in grado di scovare il plagio nei progetti software e di trovare frammenti di codice parzialmente duplicati all'interno di quest'ultimi anche se di grandi dimensioni.
		
		L'output di SIM può essere processato utilizzando script da shell per produrre istogrammi o una lista di sottomissioni sospette.
	
	\section{Siff}
		L'applicazione originale di questo strumento era trovare file simili in un sistema di grandi dimensioni ed il suo nome originario era \textit{Sif} 
		La tecnica era stata utilizzata in concomitanza con altri metodi per misurare la somiglianza nei file bytecode Java. Il programma è stato poi rinominato "\textbf{Siff}". 

		Siff era stato inizialmente progettato per confrontare un gran numero di file di testo per trovare affinità tra di loro. Siff è un tool di UNIX che può confrontare due file di testo per similarità, ma assumendo 1 secondo per ogni confronto, tutti confronti a coppie tra 5000 file richiederebbe più di 12 milioni di confronti e la CPU impiegherebbe all'incirca 5 mesi per completare il lavoro.
		
		Il nuovo algoritmo di Siff cerca di ridurre questo tempo.
		I file vengono rappresentati in una forma compatta, la "approximate fingerprint". 
		Questa fingerprint può essere confrontata velocemente, ma concede differenze nei file, solo una somiglianza del 25\%). 
		Se i due file hanno 5-10 fingerprints in comune, vengono classificati come "simili", dipenderà dalla dimensione del file. 
		Il vantaggio di questo metodo è che il fingerprint tra due file simili avrà una grande intersezione e due file non in relazione avranno invece una piccola intersezione molto probabilmente. 
		Di solito la probabilità di trovare la stessa stringa di 50 byte in due file non in relazione è molto scarsa ma il metodo è suscettibile a "cattive" fingerprints.
		Per esempio, quando si scrive un testo di piccole dimensioni è facile trovare molte somiglianze con altri file. Questo può portare a identificare simili due file non in relazione. Questo può avvenire estraendo soltanto del testo dai documenti.
		
		I vari usi di Siff possono essere: 
		
		\begin{itemize}
			\item Confrontare differenti revisioni del codice di un programma per il plagio.
			\item Trovare copie duplicate di file in un grande archivio di dati.
			\item Trovare file simili in Internet così da ridurre la ricerca in cartelle dove ci potrebbero essere file simili come quelli già visti.
			\item per gli editori - trovare plagio.
			\item Per scopi accademici - trovare plagio.
			\item Per raggruppare insieme file simile prima che vengano compressi.
			\item Confrontare revisioni di file su macchine diverse, ad esempio quelle di casa, del lavoro o portatili. Anche se quest'ultimo punto è stato superato dal controllo di versioni come Git.
		\end{itemize}
		
		
	\section{Plague}\label{def:Plague}
		\textbf{Plague} è l'implementazione dei suggerimenti trovati in  \citep{Whale1990}.
		Whale nel suo paper suggerisce le misure che possono essere usate per classificare i risultati e definisce quattro termini che possono essere usati nella sua valutazione: 
		\begin{itemize}
			\item\textbf{Recall}: il numero di documenti rilevanti consultati come una proporzione di tutti i documenti rilevanti.
			\item \textbf{Precision(p)}: la proporzione dei documenti rilevanti nel gruppo consultato.
			\item \textbf{Sensitivity}: il limite affinché una data coppia di sottomissione venga nominata "simile" (abilità nel trovare somiglianze).
			\item \textbf{Selectivity}: l'abilità di mantenere un'alta precisione con l'aumentare della sensitivity (abilità nel rifiutare le differenze).
		\end{itemize}
		
		Un sistema di rilevamento dovrebbe permettere alla sensitivity di essere aggiustata, questo è simile a un sistema di recupero testi che permette all'utente di espandere o restringere una query. Questo permette uno scambio tra il numero di documenti e le somiglianze da aggiustare. Ci dovrebbe essere un limite dove la sensitivity selezionerà correttamente tutti i documenti plagiati e scarterà quelli dissimili.
		
		Plague presenta alcuni problemi: 
		
		\begin{itemize}
			\item È difficile da far adattare a nuovi linguaggi; richiede molto tempo. 
			\item I risultati dati in output da Plague sono due liste ordinate dagli indici H e HT che hanno bisogno di essere interpretate. I risultati non sono immediatamente ovvi.
			\item Plague soffre di problemi di efficienza e fa affidamento su un numero addizionale di strumenti UNIX. Questo crea problemi di portabilità.
		\end{itemize}
		Complessivamente i risultati ottenuti sono buoni.
		
	\section{YAP}
		\textbf{YAP} ovvero "Yet Another Plague" è una serie di strumenti basata su Plague~(\ref{def:Plague}.
		Michael Wise creò la versione originale \textbf{YAP1} nel 1992 e seguì con una versione ottimizzata poco dopo chiamata \textbf{YAP2}. 
		Nel 1996 Wise produsse la versione finale di YAP (YAP3) che lavorava in maniera molto efficiente con le sequenze trasposte. 
		Lo scopo principale di YAP era di costruire sulle fondamenta di Plague, uno strumento che era ancora basato su tecniche di conteggio degli attributi e delle strutture, ma che superava alcuni dei problemi vissuti con queste tecniche. 
		
		Tutti I sistemi YAP lavorano essenzialmente nella stessa maniera e le operazioni si possono suddividere in due fasi:
		\begin{enumerate}
			\item La prima fase genera un token file per ciascuna sottomissione.
			\item La seconda fase confronta coppie di token file.
		\end{enumerate}
		
		I token rappresentano oggetti nel linguaggio scelto che possono essere sia strutture linguistiche oppure funzioni incorporate.
		Tutti gli identificatori sono costanti e vanno ignorati. 
		Un piccolo assegnamento di solito ha 100-150 token mentre uno grande ne ha 400-700. 
		La fase di tokenizzazione per ciascun linguaggio è eseguito separatamente, ma tutte le versioni hanno la stessa struttura: 
		
		\begin{enumerate}
			\item Preprocessare i report sottomessi:
				\begin{enumerate}
					\item Rimuovere i commenti e le print-string.
					\item Rimuovere le lettere non trovate negli identificatori legali.
					\item Tradurre le lettere maiuscole in minuscole.
					\item Formare una lista di token primitivi.
				\end{enumerate}
			\item Cambiare i sinonimi alla forma comune. Questo è molto simile ad associare le parole ai loro iperonimi\footnote{iperonimia, ovvero quando un termine è incluso nel significato dell'altro (es: cinquecento - utilitaria -automobile)}.
			\item Identificare blocchi di funzioni/procedure.
			\item Stampare blocchi di funzioni nell'ordine di chiamata. Blocchi di funzioni ripetuti vengono estesi una sola volta, le chiamate successive vengono rimpiazzate da un token numerato che rappresenta quella funzione. Questo previene un'esplosione nei token.
			\item Riconoscere e stampare token che rappresentano parti della lingua di destinazione da un dato vocabolario. Le chiamate di funzioni vengono mappate a FUN e altri identificatori vengono ignorati.
		\end{enumerate}
		
		YAP è capace di gestire:
		\begin{itemize}
			\item Cambiamenti di commenti o nella formattazione.
			\item Cambiamenti di identificatori.
			\item Cambiamenti nell'ordine degli operandi.
			\item Cambiamenti nei tipi di dato.
			\item Rimpiazzare espressioni con altre equivalenti. YAP anche su questo, come per i punti precenti, è immune oppure a volte registra una piccola differenza (1 token).
			\item Cambiamento nell'ordine delle statement indipendenti.
			\item Cambiamento della struttura di statement di iterazione.
			\item Cambiamento della struttura di statement di selezione.
			\item Rimpiazzare chiamate di procedura dal corpo della procedura.
			\item introduzione di statement non strutturati - questo crea 1 token di differenza.
			\item Combinare segmenti di programma originale con altri copiati.
		\end{itemize}
		
		YAP è accurato come Plague nel trovare alte somiglianze e nell'evitare di trovare risultati dove non esistono. 
		
	\section{JPlag}
		Il sistema JPlag~\citep{JPlag} trova somiglianze tra più set di file di codice sorgente. JPlag non si limita a confrontare byte di testo, ma è consapevole della sintassi del linguaggio di programmazione e della struttura del programma. Questa consapevolezza lo rende robusto contro i tentativi di mascherare le somiglianze tra i file plagiati. JPlag supporta \textit{Java}, \textit{C\#}, \textit{C}, \textit{C++}, \textit{Scheme} e il linguaggio naturale.
		
		La forza di JPlag risiede anche nella potente interfaccia grafica per presentare i suoi risultati(Figure~\ref{img:JPlag}), che risulta essere simile a quella utilizzata da MOSS~(\ref{def:MOSS}).
		
		\begin{center}
			\pgfuseimage{JPlag01}
			\pgfuseimage{JPlag02}
			\captionof{figure}{Esempio della visualizzazione dei risultati di JPlag e della pagina di analisi degli ultimi}
			\label{img:JPlag}
		\end{center}
	% !TEX root = ../tesi.tex
% !TEX encoding = UTF-8
% !TEX program = pdflatex
\chapter{Studio ed implementazione}
	\backmatter
	% !TEX root = ../tesi.tex
% !TEX encoding = UTF-8
% !TEX program = pdflatex
\chapter{Conclusioni}

	
	\appendix
	
	\bibliographystyle{plainnat}
	\bibliography{bibliography}
	\addcontentsline{toc}{chapter}{Bibliografia}
	
\end{document}
