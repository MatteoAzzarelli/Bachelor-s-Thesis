% !TEX root = ../tesi.tex
% !TEX encoding = UTF-8
% !TEX program = pdflatex

\chapter{GitHub Classroom}
	\section{GitHub}
		\textbf{GitHub} è un servizio di hosting per il controllo delle versioni basato su \textit{Git}.
		Esso, oltre al controllo delle versioni, fornisce anche altre funzionalità di collaborazione come il bug tracking, gestione delle attività e wiki per ogni progetto.
		
		GitHub offre la possibilità di creare repositories sia private che pubbliche, quest'ultime spesso sono utilizzate per condividere progetti open-source, molto importante per la comunità scientifica.
		
		Nato nel 2008 questo progetto prende subito piede, infatti in nemmeno un anno raggiunge 100.000 utenti iscritti. Dopo 10 anni dalla sua nascita arriva a 22 milioni di iscrizioni diventando uno standard  e un requisito necessario per tutti i programmatori.
		
		Un importantissima iniziativa lanciata da GitHub è il nuovo programma \textbf{GitHub Student Developer Pack} che concede gratuitamente agli studenti un insieme dei tools e servizi più popolari come ad esempio \textit{awsEducate} ,\textit{bitnami} o ancora \textit{Microsoft Azure} e ovviamente \textit{GitHub} offre repositories private illimitate.
		GitHub offre agli insegnanti il tool chiamato \textbf{GitHub Classroom}, che si propone come uno strumento per aiutare gli insegnanti ad  educare le nuove generazioni di sviluppatori ad utilizzare gli strumenti richiesti dalle aziende e a padroneggiare il linguaggio necessario per immettersi nel mondo del lavoro.
	
		\subsection{Git}
			Una breve digressione sul \textit{controllo di versione distribuito \textbf{Git}}.
			
			Esso tiene traccia dei cambiamenti dei file e coordian il lavoro su questi file tra un team di più persone.
			\'E utilizzato prevalentemente per la gestione di codice sorgente di progetti di software, ma esso può essere utilizzato per tenere traccia dei cambiamenti in  qualsiasi set di files.
			
			Pensato per essere un sistema di controllo distribuito esso è stato progettato per mantenere l'integrità dei files, di garantire un'alta velocità e anche di gestire dei flussi di lavoro non lineari.
			
			Questo magnifico software è stato creato da \textit{Linus Torvalds} nel 2005 insieme ad altri suoi colleghi per facilitare lo sviluppo del kernel di Linux .
			
			Git si distingue da tutti gli altri software di versione per il modo in cui immagazzina i dati. A differenza dei precedenti VCS che salvano i dati come una lista delle modifiche (figura~\ref{img:storingDataAsChanges}), Git salva i dati come delle istantanee (figura~\ref{img:storingDataAsSnapshot}). 
			
			\begin{center}
				\pgfuseimage{storingDataAsChanges}
				\captionof{figure}[Salvataggio dati come cambiamenti]{Gli altri VCS tendono ad immagazzinare i dati come cambiamenti alla versione base di ogni file.}
				\label{img:storingDataAsChanges}
			\end{center}
		
			\begin{center}
				\pgfuseimage{storingDataAsSnapshots}
				\captionof{figure}[GIT salvataggio dati come snapshot]{Git immagazzina i dati come snapshot del progetto nel tempo.}\label{img:storingDataAsSnapshot}
			\end{center}

		\subsection{GitHub Classroom} 
	
	

	
