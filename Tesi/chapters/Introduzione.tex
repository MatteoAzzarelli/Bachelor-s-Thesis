% !TEX root = ../tesi.tex
% !TEX encoding = UTF-8
% !TEX program = pdflatex

\chapter*{Introduzione}
\addcontentsline{toc}{chapter}{Introduzione} %aggiungo la numerazione all'indice
\chaptermark{Introduzione}

	Questa applicazione nasce dall'esigenza di trovare le similitudini tra una moltitudine di progetti di esami scritti in linguaggio \verb|C|. 
	
	Tenere sotto controllo centinaia di progetti, ricordandosi ogni singolo codice è umanamente impossibile, per questo motivo il seguente programma è un'eccellete supporto alla valutazione. Infatti il docente non ha più l'onere di memorizzare ogni codice alla ricerca di illeciti, piuttosto potrà concentrarsi sulla valutazione delle effettive capacità degli alunni.
	Lo scopo quindi è di analizzare i compiti degli studenti incrociando gli esami dello stesso appello e se necessario anche degli appelli precedenti, così da ridurre al minimo la probabilità di sfuggire al controllo.
	
	Ci auguriamo che gli studenti prendendo coscienza dell'effettiva esistenza di un software di controllo funzionante siano dissuasi dal perpetrare le azioni di copiatura. Questa teoria è anche supportata da tutti gli sviluppatori di applicazioni di plagiarism detection. 
	
	In questo progetto vedremo cosa è GitHub~(\ref{def:GitHub}) e GitHub Classroom~(\ref{def:Classroom}) con annessa introduzione al software Git~(\ref{def:Git}). Dopodiché andremo alla scoperta dei maggiori programmi di plagiarism detection~(\ref{def:PlagiarismDetection}) e in fine vedremo più in dettaglio come è stata realizzata l'applicazione~(\ref{def:StudioImplementazione}) che permette di scaricare da GitHub i progetti di esame degli studenti e in seguito di analizzare li stessi tramite il supporto di MOSS~(\ref{def:MOSS}).
