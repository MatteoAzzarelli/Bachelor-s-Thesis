% !TEX root = ../tesi.tex
% !TEX encoding = UTF-8
% !TEX program = pdflatex

\chapter*{Introduzione}
\addcontentsline{toc}{chapter}{Introduzione} %aggiungo la numerazione all'indice
\chaptermark{Introduzione}

	In questo progetto vedremo cosa è GitHub~(\ref{def:GitHub}) e GitHub Classroom~(\ref{def:Classroom}) con annessa introduzione al software Git~(\ref{def:Git}). Dopodiché andremo alla scoperta dei maggiori programmi di plagiarism detection~(\ref{def:PlagiarismDetection}) e in fine vedremo più in dettaglio come è stata realizzata l'applicazione~(\ref{def:StudioImplementazione}) che permette di scaricare da GitHub i programmi degli studenti e in seguito di analizzare li stessi tramite il supporto di MOSS~(\ref{def:MOSS}).
