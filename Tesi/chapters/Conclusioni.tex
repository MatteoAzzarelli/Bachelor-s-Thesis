% !TEX root = ../tesi.tex
% !TEX encoding = UTF-8
% !TEX program = pdflatex
\chapter{Conclusioni}
	Dopo aver visto le potenzialità del software Git e della piattaforma che offre un eccellente servizio gratuito come GitHub, siamo passati all'analisi dei vari software ideati per scovare similitudini all'interno di progetti software e scoraggiare i tentativi di copiatura.
	Dopo questo importante studio abbiamo descritto l'implementazione del programma da noi sviluppato riportando i passaggi ritenuti più interessanti. Da questo siamo passati alla descrizione del servizio indispensabile di MOSS, così da poter prendere coscienza su cosa stiamo utilizzando e fornendo un supporto alla comprensione dei risultati.
	
	In una fase di test dell'applicazione abbiamo potuto costatare l'effettivo funzionamento ed un'efficacia strabiliante nel trovare i plagi. Quindi ci riteniamo soddisfatti del lavoro fatto, ma sempre pensando a dei miglioramenti futuri.
	
	Questa è un'applicazione che può essere ampliata a piacimento e in particolare potrebbe essere utile aggiungere le seguenti features:
	\begin{itemize}
		\item Download dei risultati di MOSS (dopo 14 giorni vengono eliminati dal server di MOSS).
		\item Aggiungere una parte sull'analisi statica dei progetti (es. Valgrind per progetti in \verb|C|), per fornire aiuto al docente con una pre-valutazione del progetto.
		\item Ottenere il nome e il cognome degli studenti dal file ReadMe delle repository e costruire dinamicamente un gestionale degli studenti.
		\item ...
	\end{itemize}