% !TEX root = ../tesi.tex
% !TEX encoding = UTF-8
% !TEX program = pdflatex
\chapter{Conclusioni}
	Dopo aver visto le potenzialità del software Git e della piattaforma che offre un eccellente servizio gratuito come GitHub, siamo passati all'analisi dei vari software ideati per scovare similitudini all'interno di progetti software e scoraggiare i tentativi di copiatura.
	Dopo questo importante studio si è descritta l'implementazione del programma da noi sviluppato riportando i passaggi ritenuti più interessanti. Poi ovviamente si è descritto il servizio indispensabile di MOSS,così da poter prendere coscienza su cosa stiamo utilizzando e fornendo un supporto alla comprensione dei risultati.
	
	Ovviamente questa è un'applicazione che può essere ampliata a piacimento, e in particolare potrebbe essere utile nel futuro aggiungere le seguenti features:
	\begin{itemize}
		\item Download dei risultati di MOSS (dopo 14 giorni vengono eliminati dal server di MOSS).
		\item Aggiungere una parte sull'analisi statica dei progetti (es. Valgrind per progetti in \verb|C|).
		\item Ottenere il nome e il cognome degli studenti dal file ReadMe delle repository e costruire dinamicamente un gestionale degli studenti.
		\item ...
	\end{itemize}